\documentclass[
	a4paper, % Papierformat 
	oneside, % einseitiges Dokument 
	ngerman, % für Umlaute, Silbentrennung etc.
	12pt, % Schriftgröße
	%parskip=half, % Abstand zwischen Absätzen (halbe Zeile) 
	chapteratlists=0pt, % abstand einträge in listen (abbildungs- und tabellenverzeichnis)
	listof=totocnumbered, % Verzeichnisse im Inhaltsverzeichnis aufführen
	%index=totoc, % Index im Inhaltsverzeichnis aufführen
	numbers=noenddot, % keine Punkte hinter der letzten Zahl
	%captions=tableheading, % Beschriftung von Tabellen unterhalb ausgeben 
	%final % Status des Dokuments (final/draft)
	parskip % Zeile Abstand zwischen Absätzen.
]{scrreprt}

\usepackage{chngcntr} 
\counterwithout{footnote}{chapter}

% set title, author and date to use in titlepage
\makeatletter
\let\newtitle\@title
\let\newauthor\@author
\let\newdate\@date
\makeatother


\usepackage[utf8]{inputenc}
\usepackage[ngerman]{babel}
\usepackage[T1]{fontenc}
\usepackage{selinput}
\usepackage{graphicx}
\usepackage{lmodern}
\usepackage{blindtext}

% text spacing 
\usepackage[onehalfspacing]{setspace}

\usepackage[babel,german=quotes]{csquotes}

% bibliography style
%\usepackage[style=numeric-comp,backend=biber]{biblatex}

\usepackage[backend=biber,style=numeric-comp,citestyle=authoryear,firstinits=true,isbn=false,doi=false]{biblatex}

\addbibresource{references/references.bib}
\setcounter{biburllcpenalty}{9000}% Kleinbuchstaben
\setcounter{biburlucpenalty}{9000}% Großbuchstaben

% tables
\usepackage{longtable}
\usepackage{tabularx}
\usepackage{booktabs}

\newlength{\colthree} %definiere neue länge
\setlength{\colthree}{\textwidth} %setze neue länge auf textbreite
\addtolength{\colthree}{-6\tabcolsep} %subtrahiere -6\cdot textbreite von asdf

\newlength{\colfour} %definiere neue länge
\setlength{\colfour}{\textwidth} %setze neue länge auf textbreite
\addtolength{\colfour}{-8\tabcolsep} %subtrahiere -8\cdot textbreite von asdf

\newlength{\colfive} %definiere neue länge
\setlength{\colfive}{\textwidth} %setze neue länge auf textbreite
\addtolength{\colfive}{-10\tabcolsep} %subtrahiere -8\cdot textbreite von asdf

% images
\usepackage{transparent}
\graphicspath{{_img/}}

\SelectInputMappings{
	adieresis={ä},
	germandbls={ß},
	Euro={€}
}

% colors
\usepackage{color}
\definecolor{lightgray}{rgb}{0.9,0.9,0.9}
\definecolor{gray}{rgb}{0.5,0.5,0.5}

\usepackage[aboveskip=10pt,belowskip=0pt]{caption}

% codeblocks
\usepackage{listings} 
\lstset{
	basicstyle=\footnotesize\ttfamily,
    backgroundcolor=\color{lightgray}, 
    showstringspaces=false, 
    linewidth=\textwidth, 
    tabsize=3, 
    frame=single, 
    framerule=0pt, 
    captionpos=b,       
    numbers=none, 
    numberstyle=\scriptsize, 
    xleftmargin=0pt,
    aboveskip=20pt
}
\lstset{literate=%
	{Ö}{{\"O}}1
	{Ä}{{\"A}}1
	{Ü}{{\"U}}1
	{ß}{{\ss}}2
	{ü}{{\"u}}1
	{ä}{{\"a}}1
	{ö}{{\"o}}1
}

% hyperlinks
\usepackage[
	colorlinks=true,
	breaklinks=false,
	citecolor=black,
	linkcolor=black,
	menucolor=black,
	urlcolor=black
]{hyperref}

% seitenlayout abstände
\usepackage[left=3cm,right=2cm,top=2cm,bottom=1cm,includeheadfoot]{geometry}

% header and footer
\usepackage{fancyhdr}
\pagestyle{fancy}
\fancyhf{}
\fancyhead[l]{\slshape \leftmark}
\rfoot{\thepage}

\fancypagestyle{plain}{
	\fancyhf{} 
	\fancyfoot[OR]{\thepage} 
	\renewcommand{\headrulewidth}{0pt} 
	\renewcommand{\footrulewidth}{0pt}
} 

\usepackage[hang]{footmisc}
\setlength{\footnotemargin}{-0.8em}

\usepackage{float}

% table of content
\usepackage{tocbibind}
\setcounter{secnumdepth}{5}
\setcounter{tocdepth}{3}
\usepackage{amsmath}
\usepackage{amssymb}

% abbreviation register
\usepackage[printonlyused]{acronym}

% appendix
\renewcommand\appendix{\par 
\setcounter{chapter}{0}% 

\renewcommand\thechapter{Anhang \Alph{chapter}}% 
\renewcommand\thefigure{\Alph{chapter}}%
\renewcommand\thetable{\Alph{chapter}}%
\renewcommand{\thelstlisting}{\Alph{chapter}}}

\numberwithin{table}{chapter} 
\numberwithin{figure}{chapter}	

% einrückungen in jeder section
\usepackage{indentfirst}

% absätze und einrückungen
\parindent 0pt
\parskip 12pt

% text formatting emphasizing
\usepackage{ulem}

% horizontal lines
\renewcommand{\headrulewidth}{0.4pt}
\renewcommand{\footrulewidth}{0pt}

% pagenumbering
\newcounter{alt}

\usepackage{filecontents}
\begin{filecontents}{appendixtoc.sty}

% appendix register
\ProvidesPackage{appendixtoc}[2014/01/22 unsupported LaTeX2e package]
\RequirePackage{scrbase}[2013/12/19]% frühere Versionen unterstützen keine Sprachliste bei \providecaptionname
\RequirePackage{tocstyle}
\usetocstyle{KOMAlike}

\newenvironment*{tocconditional}[1]{%
  \expandafter\ifx\csname if@toccond@#1\expandafter\endcsname
                  \csname iftrue\endcsname
  \else
    \value{tocdepth}=-10000\relax
  \fi
  \typeout{tocdepth in `#1': \the\c@tocdepth}%
}{%
}
 
% Gleich nach dem Öffnen der toc-Datei beginnen wir den Haupt-Bereich "main":
\AtBeginDocument{%
  \addtocontents{toc}{\string\begin{tocconditional}{main}}
}
% Und der letzte Bereich endet am Ende der toc-Datei.
\BeforeClosingMainAux{%
  \addtocontents{toc}{\string\end{tocconditional}}%
}
 
% Hier können nun neue Bereiche definiert (wie man das
% macht zeigen wir gleich im Anschluss) ...
\newcommand*{\newtocconditional}[2][false]{%
  \expandafter\newif\csname if@toccond@#2\endcsname
  \csname @toccond@#2#1\endcsname
}
% ... und ein- oder ausgeschaltet werden.
% (Beispiele für die Verwendung von \settocconditional sind
% weiter unten bei der Definition von \appendixtableofcontents
% zu finden.)
\newcommand*{\settocconditional}[2]{%
  \csname @toccond@#1#2\endcsname
}
 
% Neben dem (bereits aktivierten) Hauptbereich ...
\newtocconditional[true]{main}
% ... definieren wir noch einen (noch nicht aktivierten)
% Bereich für den Anhang.
\newtocconditional{appendix}
 
% Mit dem Anhang geben wir einerseits das Anhangsverzeichnis aus,
% andererseits beenden wir den aktuellen Bereich in der toc-Datei und beginnen
% den neuen Bereich "appendix". Damit im Haupt-Inhaltsverzeichnis ein Eintrag
% für das Anhangsverzeichnis erscheint, verwenden wir \addchap und zwar noch
% bevor der letzte Bereich geschlossen wird. Wenn wir es ganz sicher machen
% wollten, müssten wir die auskommentierten Zeilen noch aktivieren. So
% verlassen wir uns einfach darauf, dass vor dem appendix-Bereich der
% main-Bereich lag.
\g@addto@macro\appendix{%
%  \addtocontents{toc}{\string\end{tocconditional}^^J
%    \string\begin{tocconditional}{main}}%
  \begingroup
    \@ifundefined{tocbasic@listhead}{% Falls \tocbasic@listhead (wird von
                               % Koma-Script-Klassen verwendet) nicht
                               % definiert ist
      \@ifundefined{chapter}{% und falls \chapter nicht definiert ist,
        \section*{\listofappendixname}% \section* verwenden
      }{% aber falls \chapter definiert ist,
        \chapter*{\listofappendixname}% \chapter* verwenden
      }%
      % und noch die Kolumnentitel passend setzen.
      \@mkboth{\csname MakeMarkcase\endcsname{\listofappendixname}}%
              {\csname MakeMarkcase\endcsname{\listofappendixname}}%
    }{% Falls \toc@heading definiert ist,
      \def\@currext{appendix}% initialisieren
      \tocbasic@listhead{\listofappendixname}% und verwenden
    }%
  \endgroup
  \addtocontents{toc}{\string\end{tocconditional}^^J
    \string\begin{tocconditional}{appendix}}%
  \appendixtableofcontents
}
 

% Jetzt definieren wir das Anhangsverzeichnis selbst als Alias für die
% toc-Datei. Dabei wird aber der Hauptbereich "main" deaktiviert und der
% Anhangsbereich "appendix" aktiviert.
\newcommand*{\appendixtableofcontents}{%
  \showtoc[{ %
    \aliastoc{\tocstyleTOC}{toc}%
    \settocconditional{main}{false}%
    \settocconditional{appendix}{true}%
  }]{toc}%
}
 
% Auch wenn man einen Anhang normalerweise nicht beenden kann, so ist es
% ggf. erwünscht, dass Literaturverzeichnis, Index etc. zwar nach den Kapiteln
% des Anhangs kommen, aber dem Hauptverzeichnis zugeordnet werden sollen. Also
% benötigen wir eine Anweisung, um in der toc-Datei den aktuellen Bereich zu
% beenden und wieder einen Hauptbereich einzuschalten:
\newcommand*{\postappendix}{%
  \addtocontents{toc}{\string\end{tocconditional}^^J%
      \string\begin{tocconditional}{main}}%
}
 
% Den Namen definieren:
\newcommand*{\listofappendixname}{Table of appendices}
\AtBeginDocument{%
  \providecaptionname{american,australien,british,canadian,english,UKenglish,USenglish}\listofappendixname{Table of appendices}%
  \providecaptionname{german,ngerman,austrian,naustrian,swissgerman,nswissgerman}\listofappendixname{Anhangverzeichnis}%
}%
\end{filecontents}
 
\usepackage{appendixtoc}

% Wir wollen das Anhangsverzeichnis im Inhaltsverzeichnis, also sorgen wir
% dafür, dass das Paket tocbasic geladen ist (auch, wenn keine
% Koma-Script-Klasse verwendet wird). Das muss unbedingt _vor_ dem Laden von
% appendixtoc passieren!
\usepackage{tocbasic}
\usepackage{appendixtoc}
\setuptoc{appendix}{totoc}% dank tocbasic geht das jetzt so einfach
