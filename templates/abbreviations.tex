\addchap{Abkürzungsverzeichnis}
\label{ch:abkuerzungsverzeichnis}

	\begin{acronym}[HATEOAS]
		\acro{API}{Application Programming Interface}
		\acro{CI}{Continuous Integration}
		\acro{CSS}{Cascading Style Sheets}
		\acro{DOM}{Document Object Model}
		\acro{HATEOAS}{Hypermedia As The Engine Of Application State}
		\acro{HTML}{Hypertext Markup Language}
		\acro{HTTP}{Hypertext Transfer Protocol}
		\acro{IANA}{Internet Assigned Numbers Authority}
		\acro{IDL}{Interface Description Language}
		\acro{IP}{Internet Protocol}
		\acro{IPS}{Internet Protocol Suite}
		\acro{IRI}{Internationalized Resource Identifier}
		\acro{JSON}{JavaScript Object Notation}
		\acro{MEAN}{MongoDB ExpressJS AngularJS NodeJS}
		\acro{MVVM}{Model-View-Viewmodel}
		\acro{NPM}{Node Package Manager}
		\acro{QoS}{Quality of Service}
		\acro{REST}{Representation State Transfer}
		\acro{RFC}{Request for Comments}
		\acro{SOA}{Service-Oriented Architecture}
		\acro{SOAP}{Simple Object Access Protocol}
		\acro{SPA}{Single Page Applikation}
		\acro{SQL}{Structured Query Language}
		\acro{TCP}{Transmission Control Protocol}
		\acro{TDD}{Test-Driven Development}
		\acro{URI}{Uniform Resource Identifier}
		\acro{URL}{Uniform Resource Locator}
		\acro{URN}{Uniform Resource Name}
		\acro{VSM}{Virtual Service Management}
		\acro{WSDL}{Web Service Description Language}
		\acro{W3C}{World Wide Web Consortium}
		\acro{XML}{Extensible Markup Language}
		\acro{WWW}{World Wide Web}
		\acro{YAML}{Yet Another Multicolumn Layout}
		\acro{XRI}{Extensible Resource Identifier}
	\end{acronym}
